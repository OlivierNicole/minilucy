\documentclass{article}

\usepackage{fontspec}

\usepackage{polyglossia}
\setmainlanguage{french}

%\usepackage{mathpartir}
\usepackage{syntax}
\setlength{\grammarindent}{4em}


\title{Rapport: compilateur du langage synchrone MiniLucy vers C}
\author{Olivier~Nicole}
\date{\today}

\begin{document}

\maketitle

\section{Introduction}

\subsection{Objet du rendu}

Le compilateur que j'ai réalisé compile vers le langage C le noyau minimal
décrit dans l'article ``Clock-directed Modular Code Generation for Synchronous
Data-flow Languages'', avec les constructions \verb/merge/, \verb/when/ sur
les types booléens, et augmenté du branchement \verb/if-then-else/ (qui n'est
qu'un sucre syntaxique pour \verb/merge/ et \verb/when/).

\subsection{Point de départ, aides et inspirations}

Je suis parti du code du lexer et du parseur du \textit{model checker}
\verb/lmoch/ comme proposé dans l'énoncé. J'ai également regardé le corrigé du
TP que vous avez donné au MPRI afin de comprendre l'algorithme du tri
topologique, bien que je l'aie réécrit à ma manière. Tout le reste du code est
écrit par moi.

J'ai échangé avec Vincent Bochet à propos des règles de typage sur les horloges,
mais nous n'avons partagé aucun code. J'ai également reçu de l'aide de la part
de Guerric Chupin pour la vérification des horloges et la génération de code.

\section{Détail des différentes passes}

\subsection{Typage}

Le typage est trivial, puisqu'il n'y a pas de polymorphisme et toutes
les variables sont déclarées avec leur type. Il s'agit donc d'une simple
vérification.

Le type d'une expression est une liste de types de base, car chaque expression
peut être considérée comme un tuple.

\subsection{Vérification des horloges}

Le langage des horloges est le suivant:
\begin{grammar}

   <ck> ::= \alpha \alt `base' \alt <ck> `on' <ident>

\end{grammar}

Les horloges associées aux expressions sont en fait une liste d'horloges, pour
la même raison qu'avec les types.

On a ajouté les variables d'horloge, de sorte à pouvoir modifier la
règle de typage des constantes avec \(\vdash c : \alpha \), ce qui
permet par exemple d'éviter d'écrire, étant donné \(x\) sur l'horloge \(i\),
\verb/(1 when i) + x/ au profit de \verb/1 + x/.

Le typage des horloges est réalisé par unification destructive.

% Peut-être à mettre au moment de la génération de code cette discussion.
À noter qu'il se peut qu'une variable ait toujours une horloge
variable après le typage. Par exemple pour ce nœud :
% Tu corrigeras la syntaxe
\begin{verbatim}
node foo() return (z : int);
var x, y : int;
let
  x = 0;
  y = x;
  z = 1;
tel;
\end{verbatim}

\verb+x+ est une constante, donc son horloge est variable, de même
pour \verb+y+ qui est une copie de \verb+x+. Si on rencontre une
équation sur une horloge variable, après la vérification des horloges,
on considère qu'elle est sur l'horloge de base.

Mon intuition est que seul du code mort peut être polymorphique en horloge une
fois l'unification terminée, étant donné que si le code est utile à la
production de la sortie, l'horloge sera unifiée avec l'horloge de base ou une
sous-horloge. Toutefois, je ne l'ai pas vérifié.

\subsection{Ordonnancement}

L'ordonnancement se résume à un tri topologique du graphe des équations et de
leur dépendances, selon l'algorithme de Kahn.

\subsection{Normalisation}

On normalise aussi les applications d'opérateurs simples tels que \verb/+/. Ce
n'est pas nécessaire pour la traduction vers C, mais j'avais commencé comme ça,
et cela peut être utile pour la traduction vers un langage de plus bas niveau
comme de l'assembleur ou LLVM.

\subsection{Génération de code objet}

La génération de code objet intermédiaire ne pose pas de problème, elle se fait
comme dans l'article. J'ai toutefois du rajouter une passe de séparation des
tuples en plusieurs équations qui ne semble pas présente dans l'article. Elle
est nécessaire pour garantir que, hors applications de nœuds, toutes les
équations ait un type de base (et pas le type d'un tuple).

\subsection{Génération de code C}

La génération de code C est assez directe. Pour chaque nœud, on
commence par générer la structure qui contient son état. La mémoire d'un nœud
contient les mémoires définies par les \verb/fby/, ainsi que les instances
d'autres nœuds utilisées et les sorties. On intègre les sorties afin de pouvoir
les extraire après l'application de la fonction \verb/step/. On aurait également
pu générer une structure de résultat pour le nœud qui aurait contenu les sorties
de ce nœud et qui aurait été renvoyé par la fonction \verb+step+, mais cela
semblait plus compliqué.

Par exemple le nœud (normalisé)~:

\begin{verbatim}
node incr (tic : bool) returns (cpt : int);
var fb : int, merge : int;
let
  merge = merge tic (1 when true(tic)) (0 when false(tic))
  fb = 0 fby cpt
  cpt = (fb) + (merge)
tel

node check (x : bool) returns (ok : bool);
var cpt : int, fb : int;
let
  cpt = incr x
  fb = 0 fby cpt
  ok = (fb) <= cpt
tel
\end{verbatim}

donne~:

\begin{verbatim}
#include <stdio.h>
#include <stdbool.h>

typedef struct {
  int cpt; int fb;

} incr_state;

void incr_reset(incr_state* self) {
  self->fb = 0;
}

void incr_step(incr_state* self, bool tic) {
  int merge;
  if(tic) {
    merge = 1;
  } else {
    merge = 0;
  }

  self->fb = self->cpt;
  self->cpt = self->fb + merge;
}

typedef struct {
  int fb; bool ok;
  incr_state o;
} check_state;

void check_reset(check_state* self) {
  incr_reset(&self->o);
  self->fb = 0;
}

void check_step(check_state* self, bool x) {
  int cpt;
  incr_step(&self->o, x);
  cpt = self->o.cpt;;
  self->fb = cpt;
  self->ok = self->fb <= cpt;
}
\end{verbatim}

Le compilateur génère un code C dans le fichier \verb/out.c/. Pour simuler
l'exécution, il est nécessaire d'y ajouter une fonction \verb/main/ à la main.

\section{Limitations}

Je n'ai pas pu implémenter autant de choses que je l'aurais souhaité dans ce
compilateur. Je n'ai pas eu le temps d'implémenter la construction \verb/reset/,
ni de coder les fonctions de test demandées par l'énoncé. Je n'ai d'ailleurs pas
bien compris ce que ces fonctions étaient censées faire. En effet, cela aurait
nécessité de coder un vérificateur de la sémantique opérationnelle des
différentes représentations intermédiaires, ce qui me semblait difficile.

Toutefois, j'ai mis dans le répertoire \verb/examples/ une dizaine de fichiers
MiniLucy qui couvrent l'ensemble de la syntaxe et des points «difficiles» de la
compilation.

Également, je n'ai pas eu le temps d'implémenter les types énumérés. Les
constructions \verb/merge/ et \verb/when/ ne fonctionnent donc que sur des types
booléens.

\end{document}
